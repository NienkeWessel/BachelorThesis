In this chapter, we will briefly look at some other extensions that could be done to our language. This will mostly be informal and `quick-and-dirty', just to give the reader an idea of how what we have done here could be expanded. 

\section{While}
It is interesting to add some form of looping, as that makes it for example possible for a program to not ever finish. As said in the introduction of this thesis, we want the compile time check to always finish, even if the program itself does not finish. 

For example, the very simple program 

\begin{minted}[linenos, frame=lines]{rust}
fn main() {
    while true {
        print!("Hello world!")
    }
}
\end{minted}

produces no error. \footnote{It does give the warning \texttt{denote infinite loops with `loop { ... }`}, but as said in the introduction, we will only be trying to model actual errors.}

If we add while-loops to our syntax, we get 

\begin{definition}
\label{statementswhile}
A statement $S$ is defined recursively by:
$$S ::= \textrm{skip} \mid S_1; S_2 \mid a: \textrm{let } x:\tau \textrm{ in } S' \mid x := e \mid \texttt{while } b \{ S' \}$$
where $e$ is an expression, $\tau$ is a type and $S_1$, $S_2$ and $S'$ are again statements.
\end{definition}

We will not formally define booleans here, but just use them. We refer the interested reader to \cite{nielson1992semantics}.


