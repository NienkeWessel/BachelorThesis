Finally, I can proudly present my bachelor thesis to you. 
Let's look at what we have in front of us (quite literally, the thesis itself).

The topic of this thesis originates from the bachelor courses ``Semantics and Correctness" and ``Computation Models". For the latter, Sander Hendrix and I did a project on the natural semantics of Rust, mostly concerning memory safety aspects that Rust claims to have. This project sparked my interests for describing the semantics of programming languages and looking into how mathematics and computing science can be combined in this field. The report that we then made formed the basis for this thesis. 

There are a lot of people that helped me along the way, and I would like to thank them, so here we go; thanks mom and dad for not getting angry at me that I was not finishing my bachelor's quicker. Thanks friends for accepting that I had to cancel some chillings in order to finally finish the thing. I would like to thank Sander Hendrix for doing the project on Rust together. I want to thank Engelbert Hubbers and Erik Barendsen for their feedback on our project.

Most importantly, I want to thank Freek Wiedijk and Marc Schoolderman for being my supervisors and helping me turn this vague idea (`something with Rust and semantics') into this thesis. And for all the feedback and time, of course!

Anyways, dear reader, have fun reading!

%In front of you is the thesis \title. This is the result of a study on the semantics and the type system of the programming language Rust. 

%

%

%Blahblah

%I hope you read this thesis with the same pleasure I made it with.

%Nienke Wessel

%Nijmegen, \today
