\documentclass{beamer}
 
\usepackage[utf8]{inputenc}
\usepackage{listings}
\usepackage{bbding} % For the check mark symbol
\usepackage{amssymb} %To denote the set of integers with correct type of letter

\usetheme{AnnArbor}
\usecolortheme{wolverine}

\title[Rust Semantics]{The Semantics of Ownership and Borrowing in the Rust Programming Language}
\author[Nienke Wessel]{Nienke Wessel\\
\small Supervisors: Freek Wiedijk \& Marc Schoolderman}
\institute{Radboud University}
\date{July 2nd, 2019}
 

\newcommand{\letterfunc}[2]{\mathcal{#1} \llbracket #2 \rrbracket}

\newcommand{\sk}[0]{\texttt{skip}}
\newcommand{\nil}[0]{\texttt{Nil}}
\newcommand{\tr}[0]{\texttt{true}}
\newcommand{\fa}[0]{\texttt{false}} 

\newcommand{\ns}[3]{\langle #1, #2 \rangle \to #3}

\newcommand{\sos}[3]{\langle #1, #2, #3 \rangle}

\newcommand{\cc}[3]{[ #1, #2, #3 ]}

\newcommand{\letm}[2]{\texttt{let }#1 : \tau \texttt{ in } #2}

\newcommand{\sosb}[4]{\langle #1, #2, #3, \mathcal{#4} \rangle}

\newcommand{\letmut}[2]{\texttt{let mut }#1 : \tau \texttt{ in } #2}

\newcommand{\ccb}[4]{[ #1, #2, #3, \mathcal{#4} ]}

\newcommand{\whi}[1]{\texttt{while } b~ \{ #1 \}}

\newcommand{\ccwhile}[5]{[ #1, #2, #3, \mathcal{#4}, #5 ]}
 
 
\begin{document}
 
\frame{\titlepage}




\begin{frame}
\frametitle{Rust}
\begin{itemize}
	\item Programming language; imperative/functional
	\item Since 2010
	\item Originally from Mozzila
	\item Current developer: The Rust Project
\end{itemize}
\pause

\begin{itemize}
	\item Focus on safety, especially safe concurrency
	\item No dangling pointers, null pointers, data races
	\item Compile time check!
	\item Ownership, moving \& borrowing
\end{itemize}
\end{frame}


\begin{frame}[fragile]
\frametitle{Ownership \& moving}

\begin{lstlisting}
let vector = vec![37, 42];
let vector2 = vector;
println!("vector[0] + vector[1] = {}", vector[0] 
         + vector[1]);
\end{lstlisting}

\texttt{error: use of moved value: `vector`}
\end{frame}


\begin{frame}[fragile]
\frametitle{Ownership \& moving}
Also a move when passing arguments to a function

\begin{lstlisting}
fn function(v : Vec<i32>) {
    // Do something
}

let vector = vec![37, 42];
function(vector);
println!("vector[0] + vector[1] = {}", vector[0] 
         + vector[1]);
\end{lstlisting}

\texttt{error: use of moved value: `vector`}
\end{frame}


\begin{frame}[fragile]
\frametitle{Borrowing}
\begin{lstlisting}
fn function(v : &Vec<i32>) {
    // Do something
}

let vector = vec![37, 42];
function(&vector);
println!("vector[0] + vector[1] = {}", vector[0]
         + vector[1]);
\end{lstlisting}

Compiles! \Checkmark
\end{frame}


\begin{frame}[fragile]
\frametitle{Mutability}

\begin{lstlisting}
let x = 0;
x = 1;
\end{lstlisting}

\texttt{error: re-assignment of immutable variable `x`}
\end{frame}



\begin{frame}[fragile]
\frametitle{Mutability}

\begin{lstlisting}
let mut x = 0;
x = 1;
\end{lstlisting}

Compiles! \Checkmark
\end{frame}

\begin{frame}
\frametitle{Rules for borrowing}
\begin{enumerate}
    \item The scope of the borrower cannot outlast the scope of the original owner
\item Only one of the following can be true at any moment:
    \begin{enumerate}
        \item there are one or more references to a non-mutable resource
        \item there is exactly one reference to a mutable resource
    \end{enumerate}
\end{enumerate}
\end{frame}


\begin{frame}[fragile]
\frametitle{Rules for borrowing}

\begin{lstlisting}
    let y=0;
    let x = & y; 
    let z = & y;
\end{lstlisting}

Compiles! \Checkmark

\pause 

\begin{lstlisting}
    let mut y=0;
    let x = &mut y; 
    let z = &mut y;
\end{lstlisting}

\texttt{error: cannot borrow `y` as mutable more than once at a time}

\end{frame}


\begin{frame}[fragile]
\frametitle{Our syntax}

$$S ::= \sk \mid S_1; S_2 \mid \letm{x}{S'} \mid x = e$$
$$e ::= x \mid i \mid e_1 + e_2 \mid \&a~e$$
$$\tau ::= \texttt{Int}$$

We split up statements such as \texttt{let x = 5} in \texttt{let x in (x = 5)}
\end{frame}


\begin{frame}[fragile]
\frametitle{Our syntax}
\begin{lstlisting}
let x = 0;
let y = x;
let z = y;
\end{lstlisting}
Becomes
\begin{lstlisting}
let x: Int in (x = 0;
    let y: Int in (y = x;
        let z: Int in (z = y;
                      )
                  )
              )
\end{lstlisting}
\end{frame}

\begin{frame}[fragile]
\frametitle{Semantics}
We will be using the same notation that was used in the Semantics \& Correctness course as much as possible.

$$\sosb{S}{L}{s}{M} \Rightarrow \sosb{S'}{L'}{s'}{M'}$$

$L$ and $L'$ are lists that function like a sort of stack/list. Push part of programs that you will want to do later. Then pop them when the current instruction is empty (\sk).

$s$ and $\mathcal{M}$ are state-like functions.
\end{frame}

\begin{frame}[fragile]
\frametitle{Semantics}
We have a state $s: \textbf{Var} \to \mathbb{Z}_{ext}$ and a function $\mathcal{M}$ that does some additional bookkeeping:
$$\mathcal{M}: \textbf{Var} \to \mathcal{P}(\{\texttt{bor}, \texttt{mut}\} \cup \textbf{Var})$$ 

$\mathcal{M}$ gives for a variable
\begin{itemize}
\item Whether the variable is mutable
\item Whether the resource of the variable has been borrowed
\item What variables it is currently borrowing from
\end{itemize}
\end{frame}


\begin{frame}[fragile]
\frametitle{Compile time check}
However, in Rust the compiler does a lot of checks, while the actual execution is not bothered with them (hence no performance loss).

We want to model this in our semantics!

Solution: a compile time checker derivation system next to the normal semantics.

$$\ccb{S}{L}{s}{M} \to \ccb{S'}{L'}{s'}{M'}$$

which eventually results in either \tr ~or \fa
\end{frame}


\begin{frame}[fragile]
\frametitle{Some examples}
\begin{lstlisting}
//s(y) = -, s(y) = -, M(x) = {}, M(y) = {}                          
let mut x: Int in (  -> s(x) = |, M(x) = {mut}
    x = 0;           -> s(x) = 0
    let y: &Int in ( -> s(y) = |
        y = & mut x; -> s(y) = 0, M(x) = {bor,mut}
                        M(y) = {x}
    )                -> s(y) = -, M(y) = {}
                        M(x) = {mut}
)                    -> s(x) = -, M(x) = {}
\end{lstlisting}
\end{frame}



\begin{frame}
\frametitle{Proving}
Proving anything generally boils down to induction.

Inductive statements:  $S_1; S_2$, $\letm{x}{S'}$

For example, 
$$\sosb{S_1; S_2}{L}{s}{M} \Rightarrow \sosb{S_1}{S_2::L}{s}{M}$$
and
$$\ccb{S_1; S_2}{L}{r}{M} \to \ccb{S_1}{S_2::L}{r}{M}$$

Use the Induction Hypothesis to prove for $S_1$ and $S_2$. 
\end{frame}



\begin{frame}
\frametitle{Safety}
Now we want to prove that this compile time checker and our semantics actually go well together.

Idea by Wright and Felleisen (1994): prove \emph{progress} and \emph{preservation}.

\vspace{0.7cm}
\textbf{Progress:} if the compile time checker says a program is o.k. (i.e results in \tr), then we can make a step (or are in the final state $\sosb{\sk}{\nil}{s}{M}$).

\vspace{0.2cm}
\textbf{Preservation:} if the compile time checker says a program is o.k. and you can move a step with the semantics, then the compile time checker will also say that the resulting program is o.k.
\end{frame}


\begin{frame}
\frametitle{What else}
We have also proven that...
\begin{itemize}
    \item ... the compile time checker always terminates.
    \item ... for the currently executing statement it does not matter what is in the list.
    \item ... in a composition, the execution of the first statement is not influenced by whatever is in the second statement.
    \item ... the semantic rules are deterministic.
\end{itemize}
\end{frame}

\end{document}

    %Supervisors toevoegen
%Versies van Rust
%Rust is een hipstertaal
%performance is goed (omdat compile time check)
%Opzoeken of er grote projecten in Rust zijn
    %Pauzes misschien weg
    %Int even in texttt
%Gelijk naar de borrowing ipv eerst door de move
    %L als tekst ook op de slides zetten
%safety = progress + preservation
%Compile checker vertellen
% Rust logo toevoegen (\titlegraphic)
