\section{Introduction}
We will now add an extra feature to our language, namely borrowing. Borrowing presents us with new challenges, as we need to make sure that the variable we borrow from is not accessible during the borrow, but becomes accessible again after the borrow. We will need to keep track of what variables we have borrowed from. If we do not do so, we cannot free them afterwards. 

\section{Syntax}
First, we need to update our syntax to reflect the possibility of borrowing. The possible statements from definition \ref{statementsmove} can remain the same. We will need to update our expressions.

\begin{definition}
\label{expressionsborrow}
An expression $e$ is defined recursively by:
$$e ::= x \mid i \mid e_1 + e_2 \mid \&a~e$$
\end{definition}

We will also need to update our types, to indicate that a variable can have an indirect type.

\begin{definition}
\label{typesborrow}
A type $\tau$ is
$$\tau ::= \textrm{Int} \mid \& \tau$$
\end{definition}

\section{Some examples}

In order to illustrate what is happening in this chapter and therefore what should be happening in the semantics, we will walk through some small pieces of example code. 

\subsection{Mutability}
To start of easy, we will first look at an example of mutability without borrowing. 

\begin{minted}[linenos, frame=lines]{rust}
let mut y=0;
y = 1;
\end{minted}

We can desugar this to our own syntax:

\begin{minted}[linenos, frame=lines]{rust}
a: {
    let mut y in 
        y = 0;
        y = 1;
   }
\end{minted}

Let's see what happens here. We will try to use the semantics of the previous chapter as much as possible. At the \texttt{let}-statement, besides initializing \texttt{x} to $\perp$, we need to somehow save that \texttt{x} is mutable. We can do that by introducing a new function besides the state, $\mathcal{M}$, which tells us whether a variable is mutable or not. For purposes that will become clear later on, we will choose to let this function return $\{\texttt{mut}\}$ if the variable is mutable and let it return $\emptyset$ if the variable is not mutable. 

Then in line 4, we assign a value to \texttt{x}, which makes that $s(x) = 0$. Then in line 5, instead of rejecting the program because \texttt{x} already has a value and thus is not $\perp$, we see that \texttt{x} is mutable and accept the assignment if \texttt{x} is either $\perp$ or an element from $\mathbb{Z}$. If \texttt{x} were not mutable, then only $\perp$ would have sufficed. 

\subsection{Nonmutable borrow}
The example of this section looks at just borrowing and no mutability. 
\begin{minted}[linenos, frame=lines]{rust}
let x = 0;
let z;
let y = &x;
z = y;
\end{minted}

If we write this in our syntax, we get: 

\begin{minted}[linenos, frame=lines]{rust}
a: {
    let x: Int in x = 0;
    b: {
        let z: &Int in 
        c: {
            let y: &Int in
            y = & x;
            z = y;
        }
    }
}
\end{minted} 

Now let's look per step at what should be happening in the semantics. At the start, we have no borrows and every variable is undefined, i.e. $-$. Then we look at the code. 

\begin{minted}[linenos, frame=lines]{rust}
a: {
    let x: Int in 
\end{minted}

Like in the previous chapter, it seems wise to keep track of what variables have been initialized and which have not been. So lets make a state again as we did in the previous chapter to keep track of the values of the variables. So we will say \texttt{x} now is $\perp$ or $s(x) = \perp$.  

\begin{minted}[linenos, frame=lines]{rust}
a: {
    let x: Int in x = 0;
\end{minted}

Obviously, we will now want \verb|x| to equal $0 \in \mathbb{Z}$. Now we move on to the next \verb|let|-statement.

\begin{minted}[linenos, frame=lines]{rust}
a: {
    let x: Int in x = 0;
    b: {
        let z: &Int in
\end{minted}

Now \verb|z| has been initiazed. So we have $s(x) = 0$ and $s(z) = \perp$. For all other variables $v$, we have $s(v) = -$. 

\begin{minted}[linenos, frame=lines]{rust}
a: {
    let x: Int in x = 0;
    b: {
        let z: &Int in 
        c: {
            let y: &Int in
\end{minted}

Now we also have $s(y) = \perp$.

\begin{minted}[linenos, frame=lines]{rust}
a: {
    let x: Int in x = 0;
    b: {
        let z: &Int in 
        c: {
            let y: &Int in
            y = & x;
\end{minted}

Now we have $s(y) = 0$ and we have $s(x) = 0$. As you might remember from the chapter about Rust, you can have multiple `readers' to one location, as long as you cannot change the content of that location (to prevent data races). 

\begin{minted}[linenos, frame=lines]{rust}
a: {
    let x: Int in x = 0;
    b: {
        let z: &Int in 
        c: {
            let y: &Int in
            y = & x;
            z = y
\end{minted}

Now the value that $y$ had will be moved to $z$. So $s(y) = -$ and $s(z) = 0$. 

\begin{minted}[linenos, frame=lines]{rust}
a: {
    let x: Int in x = 0;
    b: {
        let z: &Int in 
        c: {
            let y: &Int in
            y = & x;
            z = y
        }
\end{minted}

Now $z$ goes out of scope, so we have $s(z) = -$, while still having $s(x) = 0$. 

\begin{minted}[linenos, frame=lines]{rust}
a: {
    let x: Int in x = 0;
    b: {
        let z: &Int in 
        c: {
            let y: &Int in
            y = & x;
            z = y;
        }
    }
}
\end{minted} 

After the second \}, nothing happens, as $s(y)$ already was $-$. After the last bracket, $x$ is also again set to $-$. 

\subsection{Mutable borrow}

For the second example, we will look at some mutable borrowing. As explained in the chapter about Rust, the following program produces an error, as it makes two mutable references to the same piece of memory. This makes the code prone to data races. 

\begin{minted}[linenos, frame=lines]{rust}
let mut x=0;
let y = & mut x; 
let z = & mut x;
\end{minted}

If we write this in our syntax, we get: 

\begin{minted}[linenos, frame=lines]{rust}
a: {
    let mut x: Int in x = 0;
    b: {
        let y: &Int in y = & mut x;
        c: {
            let z: &Int in z = & mut x;
        }
    }
}
\end{minted} 

So now we will walk through the code and see where the code should not be accepted. 

\begin{minted}[linenos, frame=lines]{rust}
a: {
    let mut x: Int in x = 0;
\end{minted}

After the \texttt{let}, we have $s(x) = -$. Then after the \texttt{x = 0}, we have $s(x) = 0$. This should be surprising in the light of the previous chapter and previous example. However, it is very important to not that our \texttt{x} here is \texttt{mut}able. We will need this information inside the \texttt{let}, so we should have some way to keep track of it. Let's say we have a function $\mathcal{M}$ that tells us whether a variable is \texttt{mut} or not. So we will say $\mathcal{M}(x) = \{\texttt{mut}\}$. 

\begin{minted}[linenos, frame=lines]{rust}
a: {
    let mut x: Int in x = 0;
    b: {
        let y: &Int in y = & mut x;
\end{minted}

For \texttt{y}, the same story holds as for \texttt{x} above, but \texttt{y} is not mutable. So at the end, $s(y) = 0$ and $\mathcal{M}(y) = \emptyset$. However, we want \texttt{x} to know that it has been borrowed, so cannot be changed directly anymore and we want \texttt{y} to know where it borrowed from, so we can release the borrow after the scope of \texttt{y} ends. Therefore, we need to do some additional bookkeeping. We will say $\mathcal{M}(x) = \{\texttt{bor}, \texttt{mut}\}$ and $\mathcal{M}(y) = \{x\}$. So our $\mathcal{M}$ will have a signature of the form $\textbf{Var} \to \mathcal{P}(\{\texttt{bor}, \texttt{mut}\} \cup \textbf{Var})$.

\begin{minted}[linenos, frame=lines]{rust}
a: {
    let mut x: Int in x = 0;
    b: {
        let y: &Int in y = & mut x;
        c: {
            let z: &Int in z = & mut x;
\end{minted}

An error should occur here, as this will be the second borrow to the mutable variable \texttt{x}. This could be discovered by looking at $\mathcal{M}(x)$ and seeing that both \texttt{mut} and \texttt{bor} are in there. No new borrow could be made. 

Looking at the same program, but without the last \texttt{let}, we get the following structure:

\begin{minted}[linenos, frame=lines]{rust}
//s(y) = -, s(y) = -, M(x) = {}, M(y) = {}
a: {                          
    let mut x: Int in         //s(x) = |, M(x) = {mut}
        x = 0;                //s(x) = 0
        b: {
            let y: &Int in    //s(y) = |
                y = & mut x;  //s(y) = 0, M(x) = {bor,mut}, M(y) = {x}
        }                     //s(y) = -, M(y) = {}, M(x) = {mut}
}                             //s(x) = -, M(x) = {}
\end{minted}

Here, we can see that at the end, everything is freed. After the borrow of \texttt{x} ends, we remove the \texttt{bor}row status from our $\mathcal{M}$, as \texttt{x} now can be borrowed again. 

\subsection{Another example of mutable borrowing}
The following piece of code is incorrect and produces the error \texttt{re-assignment of immutable variable}. This is because even though \texttt{y} and \texttt{z} are mutable, \texttt{x} is not, and thus cannot be reassigned to borrow from \texttt{z}. 
\begin{minted}[linenos, frame=lines]{rust}
let mut y=0;
let mut z=1;
let x = & mut y; 
x = & mut z
\end{minted}

As discussed above, we only accept an assignment of \texttt{x} if 
\begin{itemize}[noitemsep]
    \item \texttt{x} is \texttt{mut} and has value $-$ or some value from $\mathbb{Z}$, or;
    \item \texttt{x} is not \texttt{mut} and has value $-$.
\end{itemize}

Neither is the case here, so we reject this piece of code, as does the Rust compiler. 

In order to fix this, we make \texttt{x} mutable, as was done in the following piece of code. This does compile.

\begin{minted}[linenos, frame=lines]{rust}
let mut y=0;
let mut z=1;
let mut x = & mut y; 
x = & mut z
\end{minted}

Our analysis of this piece of code results in the following: 

\begin{minted}[linenos, frame=lines]{rust}
//s(y)=s(x)=s(z)=-,M(x)=M(y)=M(z)={}
a: {let mut y            //s(y)=|, M(y)={mut}
    in y=0;              //s(y)=0
    b: {let mut z in     //s(z)=|, M(z)={mut}
        z=1;             //s(z)=1
        c: {let mut x in //s(x)=|, M(x)={mut}
            x = & mut y; //s(x)=0, M(x)={y},M(y)={mut,bor}
            x = & mut z; //s(x)=1, M(x)={y,z},M(z)={mut,bor}
        }                //s(x)=-, M(x)={},M(y)=M(z)={mut}
    }                    //s(z)=-, M(z)={}
}                        //s(y)=-, M(y)={}
\end{minted}

It might be surprising to see that after the \texttt{x} is reassigned, the borrow of \texttt{y} is not undone. However, this is the case in the Rust compiler. The compiler (version 1.22.1) does not accept a reborrow of y even after x has gotten a new value. So adding the line \texttt{let v = \& mut y} after line 8 leads to a compile error. 

Now we will look at one more example before putting what we've learned so far into use. The following piece of code passes through the compile time checker.

\begin{minted}[linenos, frame=lines]{rust}
let y=0;
let z=1;
let mut x = & y; 
x = & z
\end{minted}

We will again analyze this piece to see what happens here. 

\begin{minted}[linenos, frame=lines]{rust}
//s(y)=s(x)=s(z)=-,M(x)=M(y)=M(z)={}
a:{let y in             //s(y)=|
    y=0;                //s(y)=0
    b:{let z in         //s(z)=|
        z=1;            //s(z)=1
        c:{let mut x in //s(x)=|,M(x)={mut}
            x = & y;    //s(x)=0
            x = & z;    //s(x)=1
        }               //s(x)=-,M(x)={}
    }                   //s(z)=-
}                       //s(y)=-
\end{minted}

In comparison to the previous example, very little bookkeeping needed to be done. We only need to keep track of the values of the variables, and keep in mind that \texttt{x} is mutable. There is no need to remember which variables have been borrowed or not, since we can have zero, one or multiple borrows anyways. 

\section{Semantics: Framework}
Just as in the chapter about moving, we need some mathematical definitions. This section is dedicated to updating our previous definitions and including some new definitions. 

\subsection*{Variables and expression}
We will continue to use $\mathbb{Z}_{ext}$ in the same way we did in the previous chapter. 

\begin{definition}
We define the function $\mathcal{V}: \textbf{Exp} \to \mathcal{P}(\textbf{Var})$ recursively by:
\begin{align*}
    \mathcal{V}(i)          &= \emptyset
\\  \mathcal{V}(x)          &= \{ x \}
\\  \mathcal{V}(e_1 + e_2)  &= \mathcal{V}(e_1) \cup \mathcal{V}(e_2)
\\  \mathcal{V}(\& x)       &= \emptyset
\end{align*}
\end{definition}

\begin{definition}
We define the function $\mathcal{B}: \textbf{Exp} \to \mathcal{P}(\textbf{Var})$ recursively by:
\begin{align*}
    \mathcal{B}(i)          &= \emptyset
\\  \mathcal{B}(x)          &= \emptyset
\\  \mathcal{B}(e_1 + e_2)  &= \mathcal{V}(e_1) \cup \mathcal{V}(e_2)
\\  \mathcal{B}(\& x)       &= \{ x \} \textrm{ if } \texttt{mut} \in \mathcal{M}(x) 
\end{align*}
\end{definition}

\begin{definition}
We define the function $\mathcal{C}: \textbf{Exp} \to \mathcal{P}(\textbf{Var})$ recursively by:
\begin{align*}
    \mathcal{C}(i)          &= \{ x \}
\\  \mathcal{C}(x)          &= \emptyset
\\  \mathcal{C}(e_1 + e_2)  &= \mathcal{V}(e_1) \cup \mathcal{V}(e_2)
\\  \mathcal{C}(\& x)       &= \{ x \}
\end{align*}
\end{definition}

\begin{definition}
The evaluation function $\mathcal{A}: \textbf{Exp} \times \textbf{State} \to \mathbb{Z}_{ext}$ is defined by:
\begin{align*}
    \letterfunc{A}{i}s          &= \letterfunc{N}{i}
\\  \letterfunc{A}{x}s          &= s(x)
\\  \letterfunc{A}{e_1 + e_2}s  &= \letterfunc{A}{e_1}s + \letterfunc{A}{e_2}s \textrm{ if } \letterfunc{A}{e_1}s \in \mathbb{Z} \textrm{ and } \letterfunc{A}{e_2}s \in \mathbb{Z}
\\  \letterfunc{A}{e_1 + e_2}s  &= - \textrm{ otherwise}
\\ \letterfunc{A}{\&a~x}s       &= \letterfunc{A}{x}s
\end{align*}
\end{definition}
Notice it is the same as in the previous chapter, except for the added last line. 

\section{Semantics: Big step}
In this chapter, we will move to the small step semantics right away and skip the big step semantics. These are of similar form to the previous chapter, except that we have rules of the form $\sosb{S}{L}{s}{M} \Rightarrow \sosb{S'}{L'}{s'}{M'}$, where $S$, $S'$ are statements and $s$, $s'$ are states as defined later TODOOOO Fix official definitions. %TODO

\begin{definition} 
\label{osb}
We define the following semantic derivation rules (name on the left):

\begin{tabular}{p{5em}p{23em}p{1em}}
& & \\

[load$_{\textrm{os}}$] &
\centering$\sosb{\texttt{skip}}{I::L}{s}{M} \Rightarrow \sosb{I}{L}{s}{M}$ & \\

& & \\

[comp$_{\textrm{os}}$] &
\centering$\sosb{S_1; S_2}{L}{s}{M} \Rightarrow \sosb{S_1}{S_2::L}{s}{M}$ & \\

& & \\

[ass$_{\textrm{os}}$] &
\centering $\sosb{\texttt{x = } e}{L}{s}{M} \Rightarrow \sosb{\texttt{skip}}{L}{$& \\
& \centering $s[x \mapsto \letterfunc{A}{e}s][\mathcal{V}(e)\mapsto-]
\centering }{M[\forall \textit{y} \in B(\textit{e}), \textit{y}\mapsto M(\textit{y})\cup \{\texttt{bor} \} ] 
[\textit{x} \mapsto M(\textit{x}) \cup B(\textit{e})]}$ & \medskip\\

& & \\

[let$_{\textrm{os}}$] &
\centering $\sos{\letm{x}{S}}{L}{s} \Rightarrow \sosb{S}{(x,s(x),\mathcal{M}(x))::L}{s[x\mapsto \perp]}{M}$ & \medskip\\

& & \\

[letmut$_{\textrm{os}}$] &
\centering $\sosb{\letmut{x}{S}}{L}{s}{M} \Rightarrow \sosb{S}{(x,s(x),\mathcal{M}(x))::L}{s[x\mapsto \perp]}{M[\texttt{x}\mapsto\{\texttt{mut}\}]}$ & \medskip\\

& & \\

[set$_{\textrm{os}}$] &
\centering$\sosb{(x,v,m)}{L}{s}{M} \Rightarrow \sosb{\texttt{skip}}{L}{s[x\mapsto v]}{M[\forall \textit{y} \in M(\textit{x})\cap\textbf{Var}, \textit{y} \mapsto M(\textit{y}) \setminus \{\texttt{bor}\}][\texttt{x}\mapsto \textit{m}]}$ & \medskip\\
\end{tabular} 
%Where $\mathcal{V}(e)\mapsto-$ is an abbreviation for `for all $x \in \mathcal{V}(e)$, $x \mapsto -$'.
\end{definition} 
Note that $\sosb{\texttt{skip}}{\texttt{Nil}}{s}{M}$ has no derivation; this is the/an end state. The other definitions are mostly self-explanatory. Instead of two values in the set-rule, we now have three to also keep track of $\mathcal{M}(x)$. 

We can now look at the compile time checker. In the following definition we show the actual rules. An explanation will be given below. 

\begin{definition}
\label{compiletimecheckerb}
The \emph{compile time checker} is a derivation system that has the following rules
\begin{align*}
\ccb{\texttt{skip}}{\texttt{Nil}}{r}{M} & \to \texttt{true}  \\
\ccb{\texttt{skip}}{P::L}{r}{M}       & \to \ccb{P}{L}{r}{M}  \\
\ccb{S_1; S_2}{L}{r}{M}                 & \to \ccb{S_1}{S_2::L}{r}{M}  \\
\ccb{x=e}{L}{r}{M}                     & \to \ccb{\texttt{skip}}{L}{r[x\mapsto \star][\mathcal{V}(e) \mapsto -]}{M[\forall \textit{y} \in B(\textit{e}), \textit{y}\mapsto M(\textit{y})\cup \{\texttt{bor} \} ][\textit{x} \mapsto M(\textit{x}) \cup B(\textit{e})]} \\
                                    & \textrm{if } \texttt{mut} \in \mathcal{M}(x), r(x) = \perp \textrm{ or } r(x) = \star, \\ &  \forall y \in \mathcal{C}(e), r(y) = \star \textrm{ and } \forall y \in \mathcal{B}(e), \neg \texttt{bor} \in \mathcal{M}(e) \\
                                    & \textrm{if } \neg \texttt{mut} \in \mathcal{M}(x), r(x) = \perp,  \\ & \forall y \in \mathcal{C}(e), r(y) = \star \textrm{ and } \forall y \in \mathcal{B}(e), \neg \texttt{bor} \in \mathcal{M}(e) \\
                                    & \to \texttt{false} \textrm{ otherwise}\\
\ccb{\letm{x}{S}}{L}{r}{M} & \to \ccb{S}{(x,r(x),\mathcal{M}(x))::L}{r[x\mapsto \perp]}{M} \\
\ccb{\letmut{x}{S}}{L}{r}{M} & \to \ccb{S}{(x,r(x),\mathcal{M}(x))::L}{r[x\mapsto \perp]}{M[\texttt{x}\mapsto\{\texttt{mut}\}]} \\
\ccb{(x,v,m)}{L}{r}{M}                    & \to \ccb{\texttt{skip}}{L}{r[x \mapsto v]}{M[\forall \textit{y} \in M(\textit{x})\cap\textbf{Var}, \textit{y} \mapsto M(\textit{y}) \setminus \{\texttt{bor}\}][\texttt{x} \mapsto \textrm{m}]} 
\end{align*}
\end{definition}

The first three rules are to be assumed to be self-explanatory. As for the second rule, this one has a complicated set of conditions that should apply. ADD EXPLANATION

Of course, we again want this compile time checker to always terminate. 

\begin{theorem}
\label{terminationb}
The compile checker from definition \ref{compiletimecheckerb} always terminates.
\end{theorem}

\begin{proof}
This proof is almost completely similar to that of \ref{termination}, if we define $|\letmut{x}{S}| = |S| + 3$. We therefore omit the details.
\end{proof}


\begin{theorem}
\label{preservationb}
If $\ccb{S}{L}{r}{M} \to ^* \texttt{true}$ and $\sosb{S}{L}{s}{M} \Rightarrow \sosb{S'}{L'}{s'}{M'}$ with $r$ equivalent to $s$, then there is an $r'$ that is equivalent to $s'$ and we have $\ccb{S'}{L'}{r'}{M'} \to ^* \texttt{true}$.
\end{theorem}

\begin{proof}
We assume $\sosb{S}{L}{s}{M} \Rightarrow \sosb{S'}{L'}{s'}{M'}$. We look at the different possible rules $\Rightarrow$. 
\begin{itemize}[noitemsep]
    \item $\sosb{\texttt{skip}}{I::L}{s}{M} \Rightarrow \sos{I}{L}{s}{M}$. We assume $\ccb{\texttt{skip}}{I::L}{r}{M} \to ^* \texttt{true}$. We try to determine which derivation steps could have let to \texttt{true}. The only possible step is 
    $$\ccb{\texttt{skip}}{I::L}{r}{M}\to \ccb{I}{L}{r}{M}$$
    Then we have our $r' = r$, and obviously we also have $\ccb{I}{L}{r}{M} \to^* \texttt{true}$.
    \item $\sosb{S_1; S_2}{L}{s}{M} \Rightarrow \sosb{S_1}{S_2::L}{s}{M}$. We assume $\ccb{S_1; S_2}{L}{r}{M} \to ^* \texttt{true}$. We try to determine which derivation steps could have led to \texttt{true}. The only possible step is 
    $$\ccb{S_1; S_2}{L}{r}{M} \to \ccb{S_1}{S_2::L}{r}{M}$$
    Then we have our $r' = r$ and obviously we also have $\ccb{S_1}{S_2::L}{r}{M} \to^* \texttt{true}$.
    \item The other cases are similar. \emph{Might add them later.}
\end{itemize}
This exhausts all possible cases and hence proves our theorem.
\end{proof}


\begin{theorem}
\label{progressb}
If $\ccb{S}{L}{r}{M} ->^* \texttt{true}$, then $S = \texttt{skip}$ and $L = \texttt{Nil}$ or we have $\sosb{S}{L}{s}{M} \Rightarrow \sosb{S'}{L'}{s'}{M'}$ for some $S'$, $L'$, $s'$ and $M'$, and every $s$ equivalent to $r$.
\end{theorem}

\begin{proof}
We assume $\ccb{S}{L}{r}{M} ->^* \texttt{true}$. We look at the possible derivation steps that could have led to this form. 
\begin{itemize}[noitemsep]
    \item $\ccb{\texttt{skip}}{\texttt{Nil}}{r}{M} \to \texttt{true}$. This means $S = \texttt{skip}$ and $L = \texttt{Nil}$ and that means we are done.
    \item $\ccb{\texttt{skip}}{P::L}{r}{M} \to \ccb{P}{L}{r}{M}$. This means we have $\sosb{\texttt{skip}}{P::L}{s}{M}$ with $s$ any state equivalent to $r$. We can then apply the rule [load$_{\textrm{os}}$], to get $\sosb{\texttt{skip}}{P::L}{s}{M} \Rightarrow \sosb{P}{L}{s}{M}$, which means we are done.
    \item The other cases are similar. \emph{Might add them later.}
\end{itemize}
\end{proof}



%We will first start with defining the evaluation function $\mathcal{A}$. In order to do that, we also need to know the type of the state $s$. We can leave the evaluation function for the first three types of expressions the same as in the previous section. However, we need to think of what we want $\&a~e$ to evaluate to. There are two possible logical answers. We can let it evaluate to e itself, and thus let the result of the evaluation be a pointer of some sort in the model as well. We can also recursively calculate the evaluation of $e$ and set that as the evaluation of $\&a~e$ as well. This latter model thus does not model pointers as such. In the two sections below, both models are worked out and described. We will start with the latter model, which we will call the ``non-pointer'' model. After that, the ``pointer'' model will be described. 

%However, besides the state that assigns a value to a variable, we also need a way to keep track of what variables \verb|x| borrows from. In a program such as 
%\begin{minted}[linenos, frame=lines]{rust}
%let y;
%{
%    let x;
%    {
%        y = 5;
%        x = &y
%        //do other stuff?
%    }
%}
%\end{minted}
%we need to return ownership to \verb|y| at the end of the inner brackets. However, at the end of the brackets, we have no easy way in natural semantics to see that when \verb|x| goes out of scope, it has borrowed from \verb|y|. Natural semantics does not know what happens in the brackets. Therefore we define a second function, $\mathcal{B}$, that when passed a variable \verb|x|, returns all the variables \verb|x| borrowed from. The signature of $\mathcal{B}$ is \textbf{Var} $\to \mathcal{P}(\textbf{Var})$. That means the signature of a rule becomes 

%$$\langle S, s, \mathcal{B} \rangle \to s, \mathcal{B}$$

%The semantics for this model now become:

%\medskip
%\begin{tabular}{p{5em}p{18em}p{13em}}
%[skip$_{\textrm{ns}}$] &
%\centering$\langle$ \texttt{skip} $, s, \mathcal{B} \rangle \to s, \mathcal{B}$ & \medskip\\

%[comp$_{\textrm{ns}}$] &
%\centering \AxiomC{$\langle S_1, s, \mathcal{B} \rangle \to s', \mathcal{B}' $}
%\AxiomC{$\langle S_2, s', \mathcal{B}' \rangle \to s'', \mathcal{B}''$}
%\BinaryInfC{$\langle S_1$; $S_2, s, \mathcal{B} \rangle \to s'', \mathcal{B}''$}
%\DisplayProof \medskip& \\

%[let$_{\textrm{ns}}$] &
%\centering
%\AxiomC{$\langle S, s[x\mapsto \perp], \mathcal{B} \rangle \to s', \mathcal{B}'$}
%\UnaryInfC{$\langle a : \texttt{let x } : \tau \texttt{ in } S, s, \mathcal{B} \rangle \to s'[\mathcal{B}'(x) \mapsto s\mathcal{B}'(x)][x \mapsto s(x)], \mathcal{B}'[x \mapsto *]$}
%\DisplayProof \medskip& \\

%[ass$_{\textrm{ns}}$] &
%\centering$\langle$ \texttt{x := } e$, s \rangle \to s[x \mapsto \letterfunc{A}{e}s][ev(e)\mapsto-]$ & if $\letterfunc{A}{x}s = \perp$, $\letterfunc{A}{e} \neq \perp$ and $\letterfunc{A}{e} \neq -$\medskip\\
%\end{tabular} 

%\subsection{Pointer model}



