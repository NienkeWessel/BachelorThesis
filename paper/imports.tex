\usepackage[utf8]{inputenc}
\usepackage{fancyhdr}
\pagestyle{fancy}
%\usepackage[a4paper]{geometry}
\usepackage{graphicx}
\usepackage{caption}
\usepackage{subcaption}
\usepackage[cache=false]{minted}     % For fancy code highlighting
\usepackage[shortlabels]{enumitem}
\usepackage{mathtools} %To make align* work
\usepackage{amssymb} %To denote the set of integers with correct type of letter
\usepackage{amsthm}
\usepackage{cite}
\usepackage{hyperref}
\usepackage{natbib}


%Numbering for theorems, definitions, corollaries, etc.
\newtheorem{theorem}{Theorem}[section]
\newtheorem{corollary}{Corollary}[theorem]
\newtheorem{lemma}[theorem]{Lemma}
\newtheorem{proposition}[theorem]{Proposition}

\theoremstyle{definition}
\newtheorem{definition}{Definition}[section]
\newtheorem{infdefinition}{Informal definition}[section]

\renewcommand\qedsymbol{$\diamondsuit$}

%\usepackage{rutitlepage/rutitlepage.dtx}

\usepackage{stmaryrd} %Double brackets

\usepackage{bussproofs}

\newenvironment{scprooftree}[1]%
  {\gdef\scalefactor{#1}\begin{center}\proofSkipAmount \leavevmode}%
  {\scalebox{\scalefactor}{\DisplayProof}\proofSkipAmount \end{center} }


%For defintion of fancy A
\newcommand{\twopartdef}[3]
{
	\left\{
		\begin{array}{ll}
			#1 & \mbox{if } #2 \\
			#3 & \mbox{otherwise } 
		\end{array}
	\right.
}


\newcommand{\letterfunc}[2]{\mathcal{#1} \llbracket #2 \rrbracket}

\newcommand{\sk}[0]{\texttt{skip}}
\newcommand{\nil}[0]{\texttt{Nil}}
\newcommand{\tr}[0]{\texttt{true}}
\newcommand{\fa}[0]{\texttt{false}}

\newcommand{\asssos}[0]{[ass$_{\textrm{sos}}$]}
\newcommand{\letsos}[0]{[let$_{\textrm{sos}}$]}
\newcommand{\compsos}[0]{[comp$_{\textrm{sos}}$]}
\newcommand{\loadsos}[0]{[load$_{\textrm{sos}}$]}
\newcommand{\setsos}[0]{[set$_{\textrm{sos}}$]}
\newcommand{\assns}[0]{[ass$_{\textrm{ns}}$]}
\newcommand{\letns}[0]{[let$_{\textrm{ns}}$]}
\newcommand{\compns}[0]{[comp$_{\textrm{ns}}$]}
\newcommand{\skipns}[0]{[skip$_{\textrm{ns}}$]}

\newcommand{\ns}[3]{\langle #1, #2 \rangle \to #3}

\newcommand{\sos}[3]{\langle #1, #2, #3 \rangle}

\newcommand{\cc}[3]{[ #1, #2, #3 ]}

\newcommand{\letm}[2]{\texttt{let }#1 : \tau \texttt{ in } #2}

\newcommand{\sosb}[4]{\langle #1, #2, #3, \mathcal{#4} \rangle}

\newcommand{\letmut}[2]{\texttt{let mut }#1 : \tau \texttt{ in } #2}

\newcommand{\ccb}[4]{[ #1, #2, #3, \mathcal{#4} ]}

\newcommand{\whi}[1]{\texttt{while } b~ \{ #1 \}}

\newcommand{\ccwhile}[5]{[ #1, #2, #3, \mathcal{#4}, #5 ]}
