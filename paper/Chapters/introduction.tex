%In a world where almost everything around us uses computers it seems more than logical that we know that these computers behave like we expect them to. We don't want the bank to transfer a thousand dollars out of our account by accident or our computer with valuable documents to crash. However, most of the programs that you will use are not as verified as you would like. This thesis aims to help build the theory surrounding verification of programs from a semantics and type systems point of view. I focus on the programming language Rust, and then specifically its claims to be memory safe. I will show a way to formalize the language and will prove several stuff about it. 

The question central in this thesis is: \\
\emph{How does one formally describe ownership in Rust?}\\

In order to answer this question, several subquestions were formulated: 

\begin{itemize}[noitemsep]
    \item What is ownership in Rust?
    \begin{itemize}[noitemsep]
        \item What is it used for?
        \item What syntax is relevant to denote ownership?
        \item What are lifetimes?
        \item What do lifetimes have to do with ownership?
    \end{itemize}
    \item What grammar describes ownership relevant features best?
    \item What are the semantics of these grammar parts?
    \item What type system describes ownership best?
    \item Something with proofs
\end{itemize}

%Is this thesis useful for practical things
%Hoe verhoudt deze scriptie zich tot RustBelt


%This question is then divided in several subquestions, such as one for the syntax, one for the semantics, one for the type system.